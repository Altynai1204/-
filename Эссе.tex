\documentclass{article}
\usepackage[utf8]{inputenc} % Кодировка текста
\usepackage[russian]{babel} % Русский язык
\usepackage{graphicx} % Для вставки изображений

\title{Эссе}
\author{Алтынай Карагулова}
\date{Апрель 2024 г.}

\begin{document}

\section{План работы}

\subsection{Введение}

- **Актуальность темы:**
    - Перегруженность судебной системы.
    - Предвзятость и ошибки в работе судей.
- **Появление ИИ-судей:**
    - Потенциальное решение проблем судебной системы.
- **Цель эссе:**
    - Проанализировать преимущества и недостатки использования ИИ-судей.

\subsection{Основная часть}

\subsubsection{Аргументы в пользу ИИ-судей:}

1. **Эффективность:**
    - ИИ может обрабатывать информацию быстрее и точнее, чем люди.
2. **Объективность:**
    - ИИ не подвержен предвзятости и эмоциям, что может привести к более справедливым решениям.
3. **Согласованность:**
    - ИИ-судьи будут выносить одинаковые решения по схожим делам, что повысит предсказуемость судебной системы.
4. **Доступность:**
    - ИИ-судьи могут работать 24/7, что позволит сократить очереди и ускорить рассмотрение дел.
5. **Снижение расходов:**
    - ИИ-судьи могут быть дешевле, чем содержание судей-людей.

\subsubsection{Аргументы против ИИ-судей:}

1. **Отсутствие этики:**
    - ИИ не способен понимать моральные и этические аспекты права.
2. **Ограниченность возможностей:**
    - ИИ не может заменить человеческий суд в сложных делах, требующих оценки intentions and motivations.
3. **Непрозрачность:**
    - Алгоритмы, используемые ИИ-судьями, могут быть непрозрачными, что подрывает доверие к судебной системе.
4. **Угроза безработице:**
    - Использование ИИ-судей может привести к потере рабочих мест для судей-людей.
5. **Социальные риски:**
    - Использование ИИ в судебной системе может привести к дискриминации и другим социальным проблемам.

\subsection{Заключение}

- Подведение итогов основных аргументов "за" и "против" ИИ-судей.
- Мнение экспертов/критиков.
- Выражение собственного мнения по данному вопросу.
- Примеры использования ИИ в судебной системе.

\subsection{Список литературы}

**Книги:**

* "Судьи будущего: Как искусственный интеллект изменит правосудие" - Судья в отставке Вадим Васильев
* "Искусственный интеллект в юриспруденции: проблемы и перспективы" - Коллектив авторов под редакцией д.ю.н., проф. А.А. Черданцева
* "Этика искусственного интеллекта" - Джон Дэнахер
* "Право и искусственный интеллект" - Д.В. Маурер
* "Искусственный интеллект в судебной системе: правовые и этические аспекты" - А.В. Григорьева

**Статьи:**

* "Искусственный интеллект в судебной системе: за и против"
* "Роль искусственного интеллекта в судебной системе"
* "Искусственный интеллект и будущее правосудия"
* "ИИ-судьи: утопия или антиутопия?"
* "Может ли искусственный интеллект заменить судью?"

**Сайты:**

* Scholar.google - https://scholar.google.com/ - поисковая система по научным статьям, где можно найти релевантные материалы по теме ИИ-судей.
* ScienceDirect - https://www.sciencedirect.com/ - база данных научных статей и журналов по различным областям знаний, в том числе по искусственному интеллекту и его применению в юриспруденции.

\end{document}
